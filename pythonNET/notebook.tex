
% Default to the notebook output style

    


% Inherit from the specified cell style.




    
\documentclass[11pt]{article}

    
    
    \usepackage[T1]{fontenc}
    % Nicer default font (+ math font) than Computer Modern for most use cases
    \usepackage{mathpazo}

    % Basic figure setup, for now with no caption control since it's done
    % automatically by Pandoc (which extracts ![](path) syntax from Markdown).
    \usepackage{graphicx}
    % We will generate all images so they have a width \maxwidth. This means
    % that they will get their normal width if they fit onto the page, but
    % are scaled down if they would overflow the margins.
    \makeatletter
    \def\maxwidth{\ifdim\Gin@nat@width>\linewidth\linewidth
    \else\Gin@nat@width\fi}
    \makeatother
    \let\Oldincludegraphics\includegraphics
    % Set max figure width to be 80% of text width, for now hardcoded.
    \renewcommand{\includegraphics}[1]{\Oldincludegraphics[width=.8\maxwidth]{#1}}
    % Ensure that by default, figures have no caption (until we provide a
    % proper Figure object with a Caption API and a way to capture that
    % in the conversion process - todo).
    \usepackage{caption}
    \DeclareCaptionLabelFormat{nolabel}{}
    \captionsetup{labelformat=nolabel}

    \usepackage{adjustbox} % Used to constrain images to a maximum size 
    \usepackage{xcolor} % Allow colors to be defined
    \usepackage{enumerate} % Needed for markdown enumerations to work
    \usepackage{geometry} % Used to adjust the document margins
    \usepackage{amsmath} % Equations
    \usepackage{amssymb} % Equations
    \usepackage{textcomp} % defines textquotesingle
    % Hack from http://tex.stackexchange.com/a/47451/13684:
    \AtBeginDocument{%
        \def\PYZsq{\textquotesingle}% Upright quotes in Pygmentized code
    }
    \usepackage{upquote} % Upright quotes for verbatim code
    \usepackage{eurosym} % defines \euro
    \usepackage[mathletters]{ucs} % Extended unicode (utf-8) support
    \usepackage[utf8x]{inputenc} % Allow utf-8 characters in the tex document
    \usepackage{fancyvrb} % verbatim replacement that allows latex
    \usepackage{grffile} % extends the file name processing of package graphics 
                         % to support a larger range 
    % The hyperref package gives us a pdf with properly built
    % internal navigation ('pdf bookmarks' for the table of contents,
    % internal cross-reference links, web links for URLs, etc.)
    \usepackage{hyperref}
    \usepackage{longtable} % longtable support required by pandoc >1.10
    \usepackage{booktabs}  % table support for pandoc > 1.12.2
    \usepackage[inline]{enumitem} % IRkernel/repr support (it uses the enumerate* environment)
    \usepackage[normalem]{ulem} % ulem is needed to support strikethroughs (\sout)
                                % normalem makes italics be italics, not underlines
    

    
    
    % Colors for the hyperref package
    \definecolor{urlcolor}{rgb}{0,.145,.698}
    \definecolor{linkcolor}{rgb}{.71,0.21,0.01}
    \definecolor{citecolor}{rgb}{.12,.54,.11}

    % ANSI colors
    \definecolor{ansi-black}{HTML}{3E424D}
    \definecolor{ansi-black-intense}{HTML}{282C36}
    \definecolor{ansi-red}{HTML}{E75C58}
    \definecolor{ansi-red-intense}{HTML}{B22B31}
    \definecolor{ansi-green}{HTML}{00A250}
    \definecolor{ansi-green-intense}{HTML}{007427}
    \definecolor{ansi-yellow}{HTML}{DDB62B}
    \definecolor{ansi-yellow-intense}{HTML}{B27D12}
    \definecolor{ansi-blue}{HTML}{208FFB}
    \definecolor{ansi-blue-intense}{HTML}{0065CA}
    \definecolor{ansi-magenta}{HTML}{D160C4}
    \definecolor{ansi-magenta-intense}{HTML}{A03196}
    \definecolor{ansi-cyan}{HTML}{60C6C8}
    \definecolor{ansi-cyan-intense}{HTML}{258F8F}
    \definecolor{ansi-white}{HTML}{C5C1B4}
    \definecolor{ansi-white-intense}{HTML}{A1A6B2}

    % commands and environments needed by pandoc snippets
    % extracted from the output of `pandoc -s`
    \providecommand{\tightlist}{%
      \setlength{\itemsep}{0pt}\setlength{\parskip}{0pt}}
    \DefineVerbatimEnvironment{Highlighting}{Verbatim}{commandchars=\\\{\}}
    % Add ',fontsize=\small' for more characters per line
    \newenvironment{Shaded}{}{}
    \newcommand{\KeywordTok}[1]{\textcolor[rgb]{0.00,0.44,0.13}{\textbf{{#1}}}}
    \newcommand{\DataTypeTok}[1]{\textcolor[rgb]{0.56,0.13,0.00}{{#1}}}
    \newcommand{\DecValTok}[1]{\textcolor[rgb]{0.25,0.63,0.44}{{#1}}}
    \newcommand{\BaseNTok}[1]{\textcolor[rgb]{0.25,0.63,0.44}{{#1}}}
    \newcommand{\FloatTok}[1]{\textcolor[rgb]{0.25,0.63,0.44}{{#1}}}
    \newcommand{\CharTok}[1]{\textcolor[rgb]{0.25,0.44,0.63}{{#1}}}
    \newcommand{\StringTok}[1]{\textcolor[rgb]{0.25,0.44,0.63}{{#1}}}
    \newcommand{\CommentTok}[1]{\textcolor[rgb]{0.38,0.63,0.69}{\textit{{#1}}}}
    \newcommand{\OtherTok}[1]{\textcolor[rgb]{0.00,0.44,0.13}{{#1}}}
    \newcommand{\AlertTok}[1]{\textcolor[rgb]{1.00,0.00,0.00}{\textbf{{#1}}}}
    \newcommand{\FunctionTok}[1]{\textcolor[rgb]{0.02,0.16,0.49}{{#1}}}
    \newcommand{\RegionMarkerTok}[1]{{#1}}
    \newcommand{\ErrorTok}[1]{\textcolor[rgb]{1.00,0.00,0.00}{\textbf{{#1}}}}
    \newcommand{\NormalTok}[1]{{#1}}
    
    % Additional commands for more recent versions of Pandoc
    \newcommand{\ConstantTok}[1]{\textcolor[rgb]{0.53,0.00,0.00}{{#1}}}
    \newcommand{\SpecialCharTok}[1]{\textcolor[rgb]{0.25,0.44,0.63}{{#1}}}
    \newcommand{\VerbatimStringTok}[1]{\textcolor[rgb]{0.25,0.44,0.63}{{#1}}}
    \newcommand{\SpecialStringTok}[1]{\textcolor[rgb]{0.73,0.40,0.53}{{#1}}}
    \newcommand{\ImportTok}[1]{{#1}}
    \newcommand{\DocumentationTok}[1]{\textcolor[rgb]{0.73,0.13,0.13}{\textit{{#1}}}}
    \newcommand{\AnnotationTok}[1]{\textcolor[rgb]{0.38,0.63,0.69}{\textbf{\textit{{#1}}}}}
    \newcommand{\CommentVarTok}[1]{\textcolor[rgb]{0.38,0.63,0.69}{\textbf{\textit{{#1}}}}}
    \newcommand{\VariableTok}[1]{\textcolor[rgb]{0.10,0.09,0.49}{{#1}}}
    \newcommand{\ControlFlowTok}[1]{\textcolor[rgb]{0.00,0.44,0.13}{\textbf{{#1}}}}
    \newcommand{\OperatorTok}[1]{\textcolor[rgb]{0.40,0.40,0.40}{{#1}}}
    \newcommand{\BuiltInTok}[1]{{#1}}
    \newcommand{\ExtensionTok}[1]{{#1}}
    \newcommand{\PreprocessorTok}[1]{\textcolor[rgb]{0.74,0.48,0.00}{{#1}}}
    \newcommand{\AttributeTok}[1]{\textcolor[rgb]{0.49,0.56,0.16}{{#1}}}
    \newcommand{\InformationTok}[1]{\textcolor[rgb]{0.38,0.63,0.69}{\textbf{\textit{{#1}}}}}
    \newcommand{\WarningTok}[1]{\textcolor[rgb]{0.38,0.63,0.69}{\textbf{\textit{{#1}}}}}
    
    
    % Define a nice break command that doesn't care if a line doesn't already
    % exist.
    \def\br{\hspace*{\fill} \\* }
    % Math Jax compatability definitions
    \def\gt{>}
    \def\lt{<}
    % Document parameters
    \title{python.NET-01}
    
    
    

    % Pygments definitions
    
\makeatletter
\def\PY@reset{\let\PY@it=\relax \let\PY@bf=\relax%
    \let\PY@ul=\relax \let\PY@tc=\relax%
    \let\PY@bc=\relax \let\PY@ff=\relax}
\def\PY@tok#1{\csname PY@tok@#1\endcsname}
\def\PY@toks#1+{\ifx\relax#1\empty\else%
    \PY@tok{#1}\expandafter\PY@toks\fi}
\def\PY@do#1{\PY@bc{\PY@tc{\PY@ul{%
    \PY@it{\PY@bf{\PY@ff{#1}}}}}}}
\def\PY#1#2{\PY@reset\PY@toks#1+\relax+\PY@do{#2}}

\expandafter\def\csname PY@tok@w\endcsname{\def\PY@tc##1{\textcolor[rgb]{0.73,0.73,0.73}{##1}}}
\expandafter\def\csname PY@tok@c\endcsname{\let\PY@it=\textit\def\PY@tc##1{\textcolor[rgb]{0.25,0.50,0.50}{##1}}}
\expandafter\def\csname PY@tok@cp\endcsname{\def\PY@tc##1{\textcolor[rgb]{0.74,0.48,0.00}{##1}}}
\expandafter\def\csname PY@tok@k\endcsname{\let\PY@bf=\textbf\def\PY@tc##1{\textcolor[rgb]{0.00,0.50,0.00}{##1}}}
\expandafter\def\csname PY@tok@kp\endcsname{\def\PY@tc##1{\textcolor[rgb]{0.00,0.50,0.00}{##1}}}
\expandafter\def\csname PY@tok@kt\endcsname{\def\PY@tc##1{\textcolor[rgb]{0.69,0.00,0.25}{##1}}}
\expandafter\def\csname PY@tok@o\endcsname{\def\PY@tc##1{\textcolor[rgb]{0.40,0.40,0.40}{##1}}}
\expandafter\def\csname PY@tok@ow\endcsname{\let\PY@bf=\textbf\def\PY@tc##1{\textcolor[rgb]{0.67,0.13,1.00}{##1}}}
\expandafter\def\csname PY@tok@nb\endcsname{\def\PY@tc##1{\textcolor[rgb]{0.00,0.50,0.00}{##1}}}
\expandafter\def\csname PY@tok@nf\endcsname{\def\PY@tc##1{\textcolor[rgb]{0.00,0.00,1.00}{##1}}}
\expandafter\def\csname PY@tok@nc\endcsname{\let\PY@bf=\textbf\def\PY@tc##1{\textcolor[rgb]{0.00,0.00,1.00}{##1}}}
\expandafter\def\csname PY@tok@nn\endcsname{\let\PY@bf=\textbf\def\PY@tc##1{\textcolor[rgb]{0.00,0.00,1.00}{##1}}}
\expandafter\def\csname PY@tok@ne\endcsname{\let\PY@bf=\textbf\def\PY@tc##1{\textcolor[rgb]{0.82,0.25,0.23}{##1}}}
\expandafter\def\csname PY@tok@nv\endcsname{\def\PY@tc##1{\textcolor[rgb]{0.10,0.09,0.49}{##1}}}
\expandafter\def\csname PY@tok@no\endcsname{\def\PY@tc##1{\textcolor[rgb]{0.53,0.00,0.00}{##1}}}
\expandafter\def\csname PY@tok@nl\endcsname{\def\PY@tc##1{\textcolor[rgb]{0.63,0.63,0.00}{##1}}}
\expandafter\def\csname PY@tok@ni\endcsname{\let\PY@bf=\textbf\def\PY@tc##1{\textcolor[rgb]{0.60,0.60,0.60}{##1}}}
\expandafter\def\csname PY@tok@na\endcsname{\def\PY@tc##1{\textcolor[rgb]{0.49,0.56,0.16}{##1}}}
\expandafter\def\csname PY@tok@nt\endcsname{\let\PY@bf=\textbf\def\PY@tc##1{\textcolor[rgb]{0.00,0.50,0.00}{##1}}}
\expandafter\def\csname PY@tok@nd\endcsname{\def\PY@tc##1{\textcolor[rgb]{0.67,0.13,1.00}{##1}}}
\expandafter\def\csname PY@tok@s\endcsname{\def\PY@tc##1{\textcolor[rgb]{0.73,0.13,0.13}{##1}}}
\expandafter\def\csname PY@tok@sd\endcsname{\let\PY@it=\textit\def\PY@tc##1{\textcolor[rgb]{0.73,0.13,0.13}{##1}}}
\expandafter\def\csname PY@tok@si\endcsname{\let\PY@bf=\textbf\def\PY@tc##1{\textcolor[rgb]{0.73,0.40,0.53}{##1}}}
\expandafter\def\csname PY@tok@se\endcsname{\let\PY@bf=\textbf\def\PY@tc##1{\textcolor[rgb]{0.73,0.40,0.13}{##1}}}
\expandafter\def\csname PY@tok@sr\endcsname{\def\PY@tc##1{\textcolor[rgb]{0.73,0.40,0.53}{##1}}}
\expandafter\def\csname PY@tok@ss\endcsname{\def\PY@tc##1{\textcolor[rgb]{0.10,0.09,0.49}{##1}}}
\expandafter\def\csname PY@tok@sx\endcsname{\def\PY@tc##1{\textcolor[rgb]{0.00,0.50,0.00}{##1}}}
\expandafter\def\csname PY@tok@m\endcsname{\def\PY@tc##1{\textcolor[rgb]{0.40,0.40,0.40}{##1}}}
\expandafter\def\csname PY@tok@gh\endcsname{\let\PY@bf=\textbf\def\PY@tc##1{\textcolor[rgb]{0.00,0.00,0.50}{##1}}}
\expandafter\def\csname PY@tok@gu\endcsname{\let\PY@bf=\textbf\def\PY@tc##1{\textcolor[rgb]{0.50,0.00,0.50}{##1}}}
\expandafter\def\csname PY@tok@gd\endcsname{\def\PY@tc##1{\textcolor[rgb]{0.63,0.00,0.00}{##1}}}
\expandafter\def\csname PY@tok@gi\endcsname{\def\PY@tc##1{\textcolor[rgb]{0.00,0.63,0.00}{##1}}}
\expandafter\def\csname PY@tok@gr\endcsname{\def\PY@tc##1{\textcolor[rgb]{1.00,0.00,0.00}{##1}}}
\expandafter\def\csname PY@tok@ge\endcsname{\let\PY@it=\textit}
\expandafter\def\csname PY@tok@gs\endcsname{\let\PY@bf=\textbf}
\expandafter\def\csname PY@tok@gp\endcsname{\let\PY@bf=\textbf\def\PY@tc##1{\textcolor[rgb]{0.00,0.00,0.50}{##1}}}
\expandafter\def\csname PY@tok@go\endcsname{\def\PY@tc##1{\textcolor[rgb]{0.53,0.53,0.53}{##1}}}
\expandafter\def\csname PY@tok@gt\endcsname{\def\PY@tc##1{\textcolor[rgb]{0.00,0.27,0.87}{##1}}}
\expandafter\def\csname PY@tok@err\endcsname{\def\PY@bc##1{\setlength{\fboxsep}{0pt}\fcolorbox[rgb]{1.00,0.00,0.00}{1,1,1}{\strut ##1}}}
\expandafter\def\csname PY@tok@kc\endcsname{\let\PY@bf=\textbf\def\PY@tc##1{\textcolor[rgb]{0.00,0.50,0.00}{##1}}}
\expandafter\def\csname PY@tok@kd\endcsname{\let\PY@bf=\textbf\def\PY@tc##1{\textcolor[rgb]{0.00,0.50,0.00}{##1}}}
\expandafter\def\csname PY@tok@kn\endcsname{\let\PY@bf=\textbf\def\PY@tc##1{\textcolor[rgb]{0.00,0.50,0.00}{##1}}}
\expandafter\def\csname PY@tok@kr\endcsname{\let\PY@bf=\textbf\def\PY@tc##1{\textcolor[rgb]{0.00,0.50,0.00}{##1}}}
\expandafter\def\csname PY@tok@bp\endcsname{\def\PY@tc##1{\textcolor[rgb]{0.00,0.50,0.00}{##1}}}
\expandafter\def\csname PY@tok@fm\endcsname{\def\PY@tc##1{\textcolor[rgb]{0.00,0.00,1.00}{##1}}}
\expandafter\def\csname PY@tok@vc\endcsname{\def\PY@tc##1{\textcolor[rgb]{0.10,0.09,0.49}{##1}}}
\expandafter\def\csname PY@tok@vg\endcsname{\def\PY@tc##1{\textcolor[rgb]{0.10,0.09,0.49}{##1}}}
\expandafter\def\csname PY@tok@vi\endcsname{\def\PY@tc##1{\textcolor[rgb]{0.10,0.09,0.49}{##1}}}
\expandafter\def\csname PY@tok@vm\endcsname{\def\PY@tc##1{\textcolor[rgb]{0.10,0.09,0.49}{##1}}}
\expandafter\def\csname PY@tok@sa\endcsname{\def\PY@tc##1{\textcolor[rgb]{0.73,0.13,0.13}{##1}}}
\expandafter\def\csname PY@tok@sb\endcsname{\def\PY@tc##1{\textcolor[rgb]{0.73,0.13,0.13}{##1}}}
\expandafter\def\csname PY@tok@sc\endcsname{\def\PY@tc##1{\textcolor[rgb]{0.73,0.13,0.13}{##1}}}
\expandafter\def\csname PY@tok@dl\endcsname{\def\PY@tc##1{\textcolor[rgb]{0.73,0.13,0.13}{##1}}}
\expandafter\def\csname PY@tok@s2\endcsname{\def\PY@tc##1{\textcolor[rgb]{0.73,0.13,0.13}{##1}}}
\expandafter\def\csname PY@tok@sh\endcsname{\def\PY@tc##1{\textcolor[rgb]{0.73,0.13,0.13}{##1}}}
\expandafter\def\csname PY@tok@s1\endcsname{\def\PY@tc##1{\textcolor[rgb]{0.73,0.13,0.13}{##1}}}
\expandafter\def\csname PY@tok@mb\endcsname{\def\PY@tc##1{\textcolor[rgb]{0.40,0.40,0.40}{##1}}}
\expandafter\def\csname PY@tok@mf\endcsname{\def\PY@tc##1{\textcolor[rgb]{0.40,0.40,0.40}{##1}}}
\expandafter\def\csname PY@tok@mh\endcsname{\def\PY@tc##1{\textcolor[rgb]{0.40,0.40,0.40}{##1}}}
\expandafter\def\csname PY@tok@mi\endcsname{\def\PY@tc##1{\textcolor[rgb]{0.40,0.40,0.40}{##1}}}
\expandafter\def\csname PY@tok@il\endcsname{\def\PY@tc##1{\textcolor[rgb]{0.40,0.40,0.40}{##1}}}
\expandafter\def\csname PY@tok@mo\endcsname{\def\PY@tc##1{\textcolor[rgb]{0.40,0.40,0.40}{##1}}}
\expandafter\def\csname PY@tok@ch\endcsname{\let\PY@it=\textit\def\PY@tc##1{\textcolor[rgb]{0.25,0.50,0.50}{##1}}}
\expandafter\def\csname PY@tok@cm\endcsname{\let\PY@it=\textit\def\PY@tc##1{\textcolor[rgb]{0.25,0.50,0.50}{##1}}}
\expandafter\def\csname PY@tok@cpf\endcsname{\let\PY@it=\textit\def\PY@tc##1{\textcolor[rgb]{0.25,0.50,0.50}{##1}}}
\expandafter\def\csname PY@tok@c1\endcsname{\let\PY@it=\textit\def\PY@tc##1{\textcolor[rgb]{0.25,0.50,0.50}{##1}}}
\expandafter\def\csname PY@tok@cs\endcsname{\let\PY@it=\textit\def\PY@tc##1{\textcolor[rgb]{0.25,0.50,0.50}{##1}}}

\def\PYZbs{\char`\\}
\def\PYZus{\char`\_}
\def\PYZob{\char`\{}
\def\PYZcb{\char`\}}
\def\PYZca{\char`\^}
\def\PYZam{\char`\&}
\def\PYZlt{\char`\<}
\def\PYZgt{\char`\>}
\def\PYZsh{\char`\#}
\def\PYZpc{\char`\%}
\def\PYZdl{\char`\$}
\def\PYZhy{\char`\-}
\def\PYZsq{\char`\'}
\def\PYZdq{\char`\"}
\def\PYZti{\char`\~}
% for compatibility with earlier versions
\def\PYZat{@}
\def\PYZlb{[}
\def\PYZrb{]}
\makeatother


    % Exact colors from NB
    \definecolor{incolor}{rgb}{0.0, 0.0, 0.5}
    \definecolor{outcolor}{rgb}{0.545, 0.0, 0.0}



    
    % Prevent overflowing lines due to hard-to-break entities
    \sloppy 
    % Setup hyperref package
    \hypersetup{
      breaklinks=true,  % so long urls are correctly broken across lines
      colorlinks=true,
      urlcolor=urlcolor,
      linkcolor=linkcolor,
      citecolor=citecolor,
      }
    % Slightly bigger margins than the latex defaults
    
    \geometry{verbose,tmargin=1in,bmargin=1in,lmargin=1in,rmargin=1in}
    
    

    \begin{document}
    
    
    \maketitle
    
    

    
    \section{-\/-\/-\/-\/-\/-\/-\/-\/-\/-\/-\/-\/-\/-\/-\/-\/-\/-\/-\/-\/-\/-\/-\/-\/-\/-
网络编程-\/-\/-\/-\/-\/-\/-\/-\/-\/-\/-\/-\/-\/-\/-\/-\/-\/-\/-\/-\/-\/-\/-\/-\/-}\label{ux7f51ux7edcux7f16ux7a0b-------------------------}

\begin{itemize}
\tightlist
\item
  网络目的:数据的传输
\item
  网络数据传输是一个复杂的过程
\item
  ISO:国际标准化组织
\end{itemize}

\subsection{OSI 七层模型 -\/-\/-》》
网络通信标准化流程}\label{osi-ux4e03ux5c42ux6a21ux578b-----ux7f51ux7edcux901aux4fe1ux6807ux51c6ux5316ux6d41ux7a0b}

\begin{itemize}
\tightlist
\item
  \textbf{应用层} : 提供用户服务,具体的内容由特定程序规定
\item
  \textbf{表示层} : 数据的压缩优化加密
\item
  \textbf{会话层} : 建立应用连接,选择传输层服务
\item
  \textbf{传输层} : 提供数据传输服务,流量控制
\item
  \textbf{网络层} : 路由选择,网络互连 (路由:网络路径选择)
\item
  \textbf{链路层} : 提供链路交换,具体的消息的发送
\item
  \textbf{物理层} : 物理硬件,接口,线路网卡的规定
\item
  \textbf{osi模型优点}:将功能公开,降低了网络传输中的耦合性,每一部分完成自己的功能。\\
  可以在开发和实施的过程中各司其职。 \#\#\# cookie
\item
  高内聚:单个模块功能尽量单一
\item
  低耦合:模块之间尽量减少关联和影响
\end{itemize}

\subsection{四层模型}\label{ux56dbux5c42ux6a21ux578b}

\begin{itemize}
\tightlist
\item
  \textbf{应用层} : 应用层 表示层 会话层
\item
  \textbf{传输} : 传输层
\item
  \textbf{网络层} : 网络层
\item
  \textbf{物理链路层} : 链路层和物理层
\end{itemize}

\subsection{五层模型(tcp/ip模型)}\label{ux4e94ux5c42ux6a21ux578btcpipux6a21ux578b}

\begin{itemize}
\tightlist
\item
  \textbf{应用层} : 应用层 表示层 会话层
\item
  \textbf{传输层} : 传输层
\item
  \textbf{网络层} : 网络层
\item
  \textbf{链路层} : 链路层\\
\item
  \textbf{物理层} : 物理层
\end{itemize}

\subsection{协议(网络协议)}\label{ux534fux8baeux7f51ux7edcux534fux8bae}

\begin{itemize}
\tightlist
\item
  \textbf{在网络通信中,各方必须守的规定。包括建立什么样的连接,消息结构等}
\item
  \textbf{应用层}: TFTP HTTP DNS SMTP
\item
  \textbf{传输层}: TCP UDP
\item
  \textbf{网络层}: IP
\item
  \textbf{物理层}: IEEE
\end{itemize}

\subsection{网络基本概念}\label{ux7f51ux7edcux57faux672cux6982ux5ff5}

\begin{itemize}
\item
  \textbf{主机}: "localhost" 表示本台计算机
\item
  \textbf{网络上}: 只在本地测试使用\\
  localhost 127.0.0.1 如果想在网络上进行测试\\
  0.0.0.0 172.60.50.93
\item
  查看IP网络信息 linux系统 ifconfig win系统 ipconfig
\item ~
  \subsection{获取计算机名:}\label{ux83b7ux53d6ux8ba1ux7b97ux673aux540d}

\begin{Shaded}
\begin{Highlighting}[]
\ImportTok{import}\NormalTok{ socket}
\NormalTok{socket.gethostname()}
\OperatorTok{-------------------------------}
\CommentTok{'tedu'}
\end{Highlighting}
\end{Shaded}
\item
  通过计算机名获取地址:
\end{itemize}

\begin{Shaded}
\begin{Highlighting}[]
\NormalTok{socket.gethostbyname(}\StringTok{'tedu'}\NormalTok{)}
\OperatorTok{--------------------------------}
\CommentTok{'127.0.1.1'}
\end{Highlighting}
\end{Shaded}

\begin{center}\rule{0.5\linewidth}{\linethickness}\end{center}

\subsection{IP地址}\label{ipux5730ux5740}

\begin{itemize}
\item
  在网络上用于区分一台计算机
\item
  \textbf{IPv4}: 点分十进制 e.g. 192.168.1.72 0-255\\
  32位二进制表示
\item
  \textbf{IPv6}: 128位二进制
\item
  \textbf{网络连接测试命令}: ping www.baidu.com 或 172.60.50.92
\item
  \textbf{特殊IP}
\item
  127.0.0.1 本地测试IP
\item
  0.0.0.0 本地网卡通用IP
\item
  192.168.1.0 表示当前网段
\item
  192.168.1.1 通常是网络节点设备的IP (网关)
\item
  192.168.1.255 广播地址
\item
  \textbf{域名}: 网络服务器地址的网络名称
\item ~
  \subsection{通过域名获取服务器信息}\label{ux901aux8fc7ux57dfux540dux83b7ux53d6ux670dux52a1ux5668ux4fe1ux606f}

\begin{Shaded}
\begin{Highlighting}[]
\ImportTok{import}\NormalTok{ socket}
\NormalTok{socket.gethostbyaddr(}\StringTok{'localhost'}\NormalTok{)}
\OperatorTok{------------------------------------}
\NormalTok{(}\StringTok{'localhost'}\NormalTok{, [], [}\StringTok{'127.0.0.1'}\NormalTok{])}
\end{Highlighting}
\end{Shaded}
\item
  将点分十进制IP转换为二进制
\end{itemize}

\begin{Shaded}
\begin{Highlighting}[]
\NormalTok{socket.inet_aton(}\StringTok{'127.0.0.1'}\NormalTok{)}
\OperatorTok{------------------------------------}
\NormalTok{b}\StringTok{'}\CharTok{\textbackslash{}x7f\textbackslash{}x00\textbackslash{}x00\textbackslash{}x01}\StringTok{'}
\end{Highlighting}
\end{Shaded}

\begin{itemize}
\tightlist
\item
  将二进制转换为点分十进制IP
\end{itemize}

\begin{Shaded}
\begin{Highlighting}[]
\NormalTok{socket.inet_aton(b}\StringTok{'}\CharTok{\textbackslash{}x7f\textbackslash{}x00\textbackslash{}x00\textbackslash{}x01}\StringTok{'}\NormalTok{)}
\OperatorTok{-------------------------------------}
\CommentTok{'127.0.0.1'}
\end{Highlighting}
\end{Shaded}

\begin{Shaded}
\begin{Highlighting}[]
\ImportTok{import}\NormalTok{ socket}
\NormalTok{socket.getservbyname(}\StringTok{'mysql'}\NormalTok{)}
\OperatorTok{---------------------------------------}
\DecValTok{3306}
\end{Highlighting}
\end{Shaded}

\subsection{传输层服务}\label{ux4f20ux8f93ux5c42ux670dux52a1}

\begin{itemize}
\item
  \textbf{面向连接的传输服务 -\/-\/-》 tcp协议}
\item
  传输特征:
\item
  提供可靠的传输服务

  \begin{itemize}
  \tightlist
  \item
    可靠性表现:数据在传输过程中,无失序,无差错,无重复,无丢失。
  \end{itemize}
\item
  传输过程中有建立和断开连接的过程
\item
  三次握手: 建立数据传输两端的持续连接

  \begin{enumerate}
  \def\labelenumi{\arabic{enumi}.}
  \item
    客户端向服务器发起连接请求
  \item
    服务器收到连接请求进行确认,返回报文
  \item ~
    \subsection{客户端收到服务器确认进行连接创建}\label{ux5ba2ux6237ux7aefux6536ux5230ux670dux52a1ux5668ux786eux8ba4ux8fdbux884cux8fdeux63a5ux521bux5efa}

    \subsection{\texorpdfstring{\protect\includegraphics{https://timgsa.baidu.com/timg?image\&quality=80\&size=b9999_10000\&sec=1530870145803\&di=191019d3e5a56a65ecf52d93753b1c2d\&imgtype=0\&src=http\%3A\%2F\%2Fpic.baike.soso.com\%2Fp\%2F20140124\%2F20140124094027-1554604688.jpg}}{三次握手}}\label{ux4e09ux6b21ux63e1ux624b}
  \end{enumerate}
\item
  四次挥手: 断开连接的两端,保证数据的传输完整

  \begin{enumerate}
  \def\labelenumi{\arabic{enumi}.}
  \item
    主动方发送报文 告知被动方要断开连接
  \item
    被动方返回报文,告知收到请求,准备断开
  \item
    被动方再次发送报文给主动方,告知准备完毕可以断开
  \item ~
    \subsection{主动方发送报文进行断开}\label{ux4e3bux52a8ux65b9ux53d1ux9001ux62a5ux6587ux8fdbux884cux65adux5f00}

    \subsection{\texorpdfstring{\protect\includegraphics{https://timgsa.baidu.com/timg?image\&quality=80\&size=b9999_10000\&sec=1530869682642\&di=ae7eeca830fff1c3e138a2104f2111e8\&imgtype=jpg\&src=http\%3A\%2F\%2Fimg2.imgtn.bdimg.com\%2Fit\%2Fu\%3D4257190007\%2C24334383\%26fm\%3D214\%26gp\%3D0.jpg}}{四次挥手}}\label{ux56dbux6b21ux6325ux624b}

    适用情况:文件的上传下载,网络情况良好,需要必须保证可靠性的情况
    比如:信息聊天,文件上传下载,邮件,网页获取
  \end{enumerate}
\item
  \textbf{面向无连接的传输服务 -\/-\/-》 udp协议}
\item
  传输特征:
\item
  不保证传输的可靠性
\item
  无需建立三次握手和四次挥手的连接断开过程
\item
  消息的收发比较自由,不受其他约束
\item
  适用情况:网络情况较差,对可靠性要求不高,收发消息的两端不适合建立固定连接
\item
  比如:网络视频,群聊,发送广播、
\item
  问题总结:
\end{itemize}

\begin{enumerate}
\def\labelenumi{\arabic{enumi}.}
\tightlist
\item
  osi模型问题
\item
  三次握手四次挥手问题
\item
  tcp和udp的区别
\end{enumerate}

\subsection{socket套接字编程}\label{socketux5957ux63a5ux5b57ux7f16ux7a0b}

\begin{itemize}
\tightlist
\item
  目的:通过编程语言提供的函数接口进行组合,更简单的完成基于tcp和udp通信的网络编程
\item
  套接字: 完成上述目标的编程方法方案
\end{itemize}

    \begin{Verbatim}[commandchars=\\\{\}]
{\color{incolor}In [{\color{incolor} }]:} 
\end{Verbatim}



    % Add a bibliography block to the postdoc
    
    
    
    \end{document}
